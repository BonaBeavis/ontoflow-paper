%%
%% This is file `sample-sigconf.tex',
%% generated with the docstrip utility.
%%
%% The original source files were:
%%
%% samples.dtx  (with options: `sigconf')
%% 
%% IMPORTANT NOTICE:
%% 
%% For the copyright see the source file.
%% 
%% Any modified versions of this file must be renamed
%% with new filenames distinct from sample-sigconf.tex.
%% 
%% For distribution of the original source see the terms
%% for copying and modification in the file samples.dtx.
%% 
%% This generated file may be distributed as long as the
%% original source files, as listed above, are part of the
%% same distribution. (The sources need not necessarily be
%% in the same archive or directory.)
%%
%%
%% Commands for TeXCount
%TC:macro \cite [option:text,text]
%TC:macro \citep [option:text,text]
%TC:macro \citet [option:text,text]
%TC:envir table 0 1
%TC:envir table* 0 1
%TC:envir tabular [ignore] word
%TC:envir displaymath 0 word
%TC:envir math 0 word
%TC:envir comment 0 0
%%
%%
%% The first command in your LaTeX source must be the \documentclass command.
\documentclass[sigconf]{acmart}

%%
%% \BibTeX command to typeset BibTeX logo in the docs
\AtBeginDocument{%
  \providecommand\BibTeX{{%
    \normalfont B\kern-0.5em{\scshape i\kern-0.25em b}\kern-0.8em\TeX}}}

%% Rights management information.  This information is sent to you
%% when you complete the rights form.  These commands have SAMPLE
%% values in them; it is your responsibility as an author to replace
%% the commands and values with those provided to you when you
%% complete the rights form.
\setcopyright{acmcopyright}
\copyrightyear{2021}
\acmYear{2021}
\acmDOI{}

%% These commands are for a PROCEEDINGS abstract or paper.
\acmConference[K-Cap '21]{K-Cap '21: The Eleventh International Conference on Knowledge Capture}{December 02--03, 2021}{Virtual Conference}


%%
%% Submission ID.
%% Use this when submitting an article to a sponsored event. You'll
%% receive a unique submission ID from the organizers
%% of the event, and this ID should be used as the parameter to this command.
%%\acmSubmissionID{123-A56-BU3}

%%
%% The majority of ACM publications use numbered citations and
%% references.  The command \citestyle{authoryear} switches to the
%% "author year" style.
%%
%% If you are preparing content for an event
%% sponsored by ACM SIGGRAPH, you must use the "author year" style of
%% citations and references.
%% Uncommenting
%% the next command will enable that style.
%%\citestyle{acmauthoryear}

%%
%% end of the preamble, start of the body of the document source.
\begin{document}

%%
%% The "title" command has an optional parameter,
%% allowing the author to define a "short title" to be used in page headers.
\title{OntoFlow : Easy Ontology Development Workflows for Non-technical Domain Experts}

%%
%% The "author" command and its associated commands are used to define
%% the authors and their affiliations.
%% Of note is the shared affiliation of the first two authors, and the
%% "authornote" and "authornotemark" commands
%% used to denote shared contribution to the research.
\author{Gordian Dziwis}
\authornote{}
\email{dziwis@infai.org}
\author{Lisa Wenige}
\authornotemark[1]
\email{wenige@infai.org}
\affiliation{%
  \institution{Institute for Applied Informatics}
  \streetaddress{Goerdelerring 9}
  \city{Leipzig}
  \state{Saxony}
  \country{Germany}
  \postcode{04177}
}

%%
%% By default, the full list of authors will be used in the page
%% headers. Often, this list is too long, and will overlap
%% other information printed in the page headers. This command allows
%% the author to define a more concise list
%% of authors' names for this purpose.
\renewcommand{\shortauthors}{Dziwis and Wenige et al.}

%%
%% The abstract is a short summary of the work to be presented in the
%% article.
\begin{abstract}
For many years, the development of widely applicable and high quality ontologies has been an ongoing research topic. Among the many challenges, the lack of integrated development environments for non-technical domain experts has been one of the most pressing research challenges. But while the participation of domain experts is vital for the applicability of ontologies, there are hardly any software tools available that facilitate their active engagement. We present a solution that addresses this research gap by automating the ontology development process with the help of a workflow engine. We define a pipeline that facilitates ontology implementation, serialization, documentation and testing within the scope of a seamless automatic routine than can be easily triggered by an ontology laymen with basic knowledge of bash usage. Thus, the processing pipeline takes care of most of the operations that usually have to be taken care of by an ontology or software engineer. We demonstrate the applicability of our approach for a wide range of ontologies and provide additional results on the quality level of ontologies throughout the Semantic Web landscape. 
\end{abstract}

%%
%% The code below is generated by the tool at http://dl.acm.org/ccs.cfm.
%% Please copy and paste the code instead of the example below.
%%
\begin{CCSXML}
<ccs2012>
 <concept>
  <concept_id>10010520.10010553.10010562</concept_id>
  <concept_desc>Computer systems organization~Embedded systems</concept_desc>
  <concept_significance>500</concept_significance>
 </concept>
 <concept>
  <concept_id>10010520.10010575.10010755</concept_id>
  <concept_desc>Computer systems organization~Redundancy</concept_desc>
  <concept_significance>300</concept_significance>
 </concept>
 <concept>
  <concept_id>10010520.10010553.10010554</concept_id>
  <concept_desc>Computer systems organization~Robotics</concept_desc>
  <concept_significance>100</concept_significance>
 </concept>
 <concept>
  <concept_id>10003033.10003083.10003095</concept_id>
  <concept_desc>Networks~Network reliability</concept_desc>
  <concept_significance>100</concept_significance>
 </concept>
</ccs2012>
\end{CCSXML}

\ccsdesc[500]{Computer systems organization~Embedded systems}
\ccsdesc[300]{Computer systems organization~Redundancy}
\ccsdesc{Computer systems organization~Robotics}
\ccsdesc[100]{Networks~Network reliability}

%%
%% Keywords. The author(s) should pick words that accurately describe
%% the work being presented. Separate the keywords with commas.
\keywords{ontologies, workflows, IDE, quality assurance}

%%
%% This command processes the author and affiliation and title
%% information and builds the first part of the formatted document.
\maketitle

\section{Introduction}



\section{Related Work}
Continuous integration strategies have become an indispensable part of modern software engineering.
They largely consist of clean up operations, compilation of executables, application of automated tests, and the deployment of the finished application, including generation of appropriate documentation if necessary. Part of these processes is the continuous checking of any updates/ new versions as well as the triggering of toubleshooting activities if problems occur \cite{fowler}. 
This ensures that software solutions are always up-to-date and that applications meet predefined quality standards.

CI should be applicable to dataset-specific operations. First attempts have been undertaken.
\cite{cirulli, klimek, kucera, meissner, rojas, roman, stadler}

The same holds true for ontology development. Typically, CI mechanisms are instance data collections the result of operations such as crawling, linking, or data transformation. 
. These processes are usually executed automatically and, aside from the effort of creating specifications and linking steps, are little interrupted by user interactions and manual intervention. If people work with appropriate software tools in this context, they are usually Data Scientists or Software Engineers. 

\section{Evaluation}
\subsection{Entire setup}
%datasets:
%-dataset 1:  Chowlk XML collection: https://github.com/oeg-upm/Chowlk/tree/webservice/data
%-dataset 2: All ontologies from Linked Open Vocabularies (LOV), ca. 700, dumps via
%https://lov.linkeddata.es/dataset/lov/sparql
% for each of the ontologies
%=>run the entire workflow
%=>log the processing time
%log success/failure
%have a look at each processing step individually and answer the following
%questions
%how many workflows failed here?
%what was the average processing time for this workflow step?
%STEP 1: TURTLE GENERATION (dataset 1 only)
%STEP 2: DOCUMENTATION (dataset 1 & 2)
%STEP 3: SERIALIZATION (dataset 1 & 2)
%STEP 4: ONTOLOGY TESTS (dataset 1 & 2)
%for dataset 4 only: evaluate ontology tests:
%how many schema violations (owl,rdfs) where there? (rdfunit flag -s)
%how many metadata errors where there:
%=> W3C Reference:
%"One common set of additional tags that could reasonably be included here are some of the standard Dublin Core metadata tags. The subset includes those that take simple types or strings as values. Examples include Title, Creator, Description, Publisher, and Date" (W3c OWL) + owl:versionInfo

\subsection{Results}
\bibliographystyle{ACM-Reference-Format}
\bibliography{ref}
\end{document}
\endinput
%%
%% End of file `sample-sigconf.tex'.
